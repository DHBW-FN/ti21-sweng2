\chapter{Systemidee}\label{ch:systemidee}

Im Rahmen der Erweiterung der existierenden Anwendung “BachelorChecker” soll anhand dieser \ac{OOA} das in Auftrag gegebene System \ac{AiPC} konzipiert werden.


\section{Zielbestimmung}\label{sec:zielbestimmung}

Die Integration dieser Erweiterung in die existierende Anwendung wird die Genauigkeit der Ergebnisse und damit die Trefferquote bei der Identifizierung von Plagiaten in Bachelorarbeiten durch künstliche Intelligenz erhöhen.
Dabei soll es nicht nur möglich sein, exakt übereinstimmende Textpassagen zu identifizieren, sondern auch solche, die paraphrasiert wurden.
Zudem kann durch die Erweiterung ein Wert, der den prozentualen Anteil der Übereinstimmung zwischen dem Text der Bachelorarbeit und dem der Quelle angibt, berechnet werden.


\section{Konzept}\label{sec:konzept}

Entsprechend dieser Anforderung schlägt der Auftragnehmer die im Folgenden beschriebene Softwarelösung vor, welche die Plagiatsprüfung über eine chatbasierte KI realisiert.

Die konzipierte Lösung stellt dem BachelorChecker eine Schnittstelle bereit, über die dieser nicht gesperrte \ac{PDF}-Dateien sowie zusätzliche Informationen als \ac{JSON}-Datei senden kann.
Die empfangenen Daten werden vom \ac{AiPC}-System aufbereitet und in ein geeignetes Prompt für die Verarbeitung durch die chatbasierte \ac{KI} umgewandelt.
Diese führt die Plagiatsprüfung auf Basis der im Prompt bereitgestellten Daten durch.
Das Ergebnis dieser Überprüfung wird von der \ac{KI} in einem JSON-Format mit einer Liste der übereinstimmenden Literaturen und dem prozentualen Übereinstimmungsgrad je Literatur an den \ac{AiPC} zurückgegeben.
Dort wird das resultierende \ac{JSON}-Objekt validiert und weiter aufbereitet, um dann dem BachelorChecker bereitgestellt zu werden.
Falls die Plagiatsprüfung nicht innerhalb eines festgelegten Zeitraums abgeschlossen werden kann, wird dem Auftraggeber eine alternative Möglichkeit zur Übergabe gegeben.


\section{Voraussetzungen an den BachelorChecker}\label{sec:voraussetzungen-an-den-bachelorchecker}

Es wird erwartet, dass der BachelorChecker in der Lage ist, \ac{HTTP}-Anfragen an die Schnittstelle des \ac{AiPC} zu senden.
Dabei übermittelt er eine nicht gesperrte \ac{PDF}-Datei sowie zusätzliche Informationen in Form einer \ac{JSON}-Datei.
Des Weiteren soll der BachelorChecker in der Lage sein, die resultierenden \ac{HTTP}-Antworten zu verarbeiten, die die Ergebnisse der Plagiatsprüfung enthalten.

Zusätzlich wird erwartet, dass der Auftraggeber eine angemessene Infrastruktur bereitstellt, auf der die Softwarelösung \ac{AiPC} betrieben werden kann.
Es ist ebenfalls erforderlich sicherzustellen, dass diese Infrastruktur in der Lage ist, effektiv mit der chatbasierten \ac{KI} zu kommunizieren.


\section{Entwicklungsstufen}\label{sec:entwicklungsstufen}

\subsection{MVP}\label{subsec:mvp}

Die erste Version des \ac{AiPC} soll ein \ac{MVP} darstellen.
Dafür soll der Empfang und die Validierung der Daten, die Generierung eines Prompts sowie die Aufbereitung und anschließende Bereitstellung der Ergebnisse wie oben beschrieben umgesetzt werden.
Diese erste Entwicklungsstufe wird nochmals unterteilt in einzelne Meilensteine, die jeweils eine Teilfunktionalität des Systems implementieren.

\subsubsection{Meilenstein 1: Schnittstellenimplementierung}
In diesem ersten Meilenstein soll die Schnittstelle des \ac{AiPC} implementiert werden.
Diese soll in der Lage sein, die von dem BachelorChecker gesendeten Daten zu empfangen und an die weiterverarbeitenden Komponenten weiterzuleiten.

\subsubsection{Meilenstein 2: Generierung eines Prompts}
In diesem Meilenstein soll die Generierung eines Prompts implementiert werden.
Zuvor werden die von der Schnittstelle erhaltenen Daten validiert und aufbereitet.
Das Prompt wird anschließend an die chatbasierte \ac{KI} weitergeleitet.

\subsubsection{Meilenstein 3: Aufbereitung und Bereitstellung der Ergebnisse}
In diesem Meilenstein soll die Aufbereitung und Bereitstellung der Ergebnisse implementiert werden.
Dazu werden die von der \ac{KI} erhaltenen Ergebnisse validiert und aufbereitet.
Anschließend werden die Ergebnisse an den BachelorChecker zurückgegeben.

\subsection{Mögliche Erweiterungen}\label{subsec:moegliche-erweiterungen}

\subsubsection{Erweiterung der Funktionalität}\label{subsubsec:-erweiterung-der-funktionalitat}
Eine mögliche Erweiterung kann die Stapelverarbeitung mehrerer Dateien gleichzeitig sein.
Außerdem kann die \ac{JSON}-Ausgabedatei mit weiteren Informationen angereichert werden, um detailliertere Ergebnisse über die Übereinstimmungen zu erhalten.

\subsubsection{Erweiterung der KI}\label{subsubsec:-erweiterung-der-ki}
Im Rahmen dieser Erweiterung kann eine fortschrittlichere \ac{KI} und Mechanismen zur kontinuierlichen Aktualisierung der \ac{KI}-Modelle für genauere Plagiatsprüfungen integriert werden.
Die verbesserte \ac{KI} soll zudem in der Lage sein, \ac{KI} generierte Texte zu erkennen und zu kennzeichnen.
Im Zuge dieser Erweiterung können auch weitere Sprachen für die Plagiatsprüfung unterstützt werden.


\section{Kosten}\label{sec:kosten}

Die Kosten für ein \ac{MVP} errechnen sich aus den Entwicklungskosten für das \ac{AiPC}-System.
Das Entwickler-Team besteht aus vier professionellen Softwareentwicklern.

\subsection{Kosten für die Entwicklung des AiPC}\label{subsec:kosten-fuer-die-entwicklung-des-aipc}

Für die Entwicklung wird ein Stundensatz von 180€ berechnet. % Quelle: https://www.digital-experts.com/blog/softwareentwicklung-kosten-was-kostet-eine-individualsoftware
Dabei sind sowohl der Gewinn als auch die anfallenden Steuern enthalten.
Der Arbeitsaufwand wird auf etwa 240 Personentage geschätzt.
Somit belaufen sich die Kosten für die Entwicklung von \ac{AiPC} auf 43.200€.

\subsection{Kosten für die chatbasierte KI}\label{subsec:kosten-fuer-die-chatbasierte-ki}

% Ist das schon wieder eine Technologieentscheidung?

\begin{table}[H]
    \begin{tabular}{lll}
        Model              & Input                & Output               \\
        gpt-3.5-turbo-1106 & \$0.0010 / 1K tokens & \$0.0020 / 1K tokens
    \end{tabular}
\end{table}
