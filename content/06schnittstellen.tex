\chapter{Schnittstellen zu Fremdsystemen}\label{ch:schnittstellen}


\section{Schnittstelle zu BachelorChecker}\label{sec:schnittstelle-bachelorchecker}

\begin{table}[H]
    \label{tab:schnittstelle-bachelorchecker}
    \begin{tabularx}{\textwidth}{|l|X|}
        \hline
        \textbf{Art}         & Datenschnittstelle von/zu externem System \\
        \hline
        \textbf{Beschreibung} & Die Schnittstelle zu BachelorChecker dient dem Kunden dazu, die Bachelorarbeit und JSON-Datei mit Zusatzinformationen an AiPlagiarismChecker zu übergeben.
        Auch dient die Schnittstelle der Übertragung einer JSON-Datei mit dem Ergebnis der Überprüfung zurück an BachelorChecker. \\
        \hline
        \textbf{Komplexität} & normal                                    \\
        \hline
        \textbf{Hinweise zur Verwendung} & Es wird nur ein Endpunkt angeboten und es erfolgt durch den AiPlagiarismChecker keine Authentifizierung.
        Die übertragene PDF muss nicht-gesperrt sein und die JSON-Datei muss ein noch zu bestätigendes Schema erfüllen.
        \autoref{lst:json-zusatzinfo} zeigt das Schema wie es vom potenziellen Auftragnehmer vorgeschlagen wird. \\
        \hline
    \end{tabularx}
    \caption {Schnittstelle zu BachelorChecker}
\end{table}

\subsection{Aufbau der JSON-Datei mit Zusatzinformationen}\label{subsec:json-zusatzinfo}
\begin{lstlisting}[caption={JSON-Datei mit Zusatzinformationen},captionpos=b,label={lst:json-zusatzinfo}]
{
  "$schema": "http://json-schema.org/draft-07/schema#",
  "type": "object",
  "properties": {
    "author": {
      "type": "string"
    },
    "language": {
      "type": "string",
      "enum": [
        "en",
        "de",
        "fr",
        "es",
        "it",
        "other"
      ]
    },
    "title": {
      "type": "string"
    },
    "sources": {
      "type": "array",
      "items": {
        "type": "string",
        "pattern": "^[0-9]{13}$"
      }
    }
  },
  "required": [
    "author",
    "language",
    "title"
  ]
}
\end{lstlisting}

Erklärung der Attribute:
\begin{itemize}
    \item author: Name des Autors der Bachelorarbeit
    \item language: Sprache der Bachelorarbeit
    \item title: Titel des Dokuments
    \item sources: Liste mit ISBN Nummern von Quellen, die bei der Prüfung ignoriert werden sollen
\end{itemize}

\subsection{Aufbau der JSON-Datei mit einem Ergebnis}\label{subsec:json-ergebnis}
\begin{lstlisting}[caption={JSON-Datei mit einem Ergebnis},captionpos=b,label={lst:json-ergebnis}]
{
  "$schema": "http://json-schema.org/draft-07/schema#",
  "type": "object",
  "properties": {
    "author": {
      "type": "string"
    },
    "resultSummary": {
      "type": "string"
    },
    "totalPlagiarismCoverage": {
      "type": "number",
      "minimum": 0,
      "maximum": 1
    },
    "plagiarismSources": {
      "type": "array",
      "items": {
        "type": "object",
        "properties": {
          "isbn": {
            "type": "integer"
          },
          "plagiarismLength": {
            "type": "integer"
          },
          "occurrences": {
            "type": "array",
            "items": {
              "type": "object",
              "properties": {
                "startChar": {
                  "type": "integer"
                },
                "endChar": {
                  "type": "integer"
                },
                "certainty": {
                  "type": "number",
                  "minimum": 0,
                  "maximum": 1
                }
              },
              "required": [
                "startChar",
                "endChar"
              ]
            }
          }
        },
        "required": [
          "isbn",
          "coverage",
          "occurrences"
        ]
      }
    }
  },
  "required": [
    "author",
    "resultSummary",
    "totalPlagiarismCoverage"
  ]
}
\end{lstlisting}

Erklärung der Attribute:
\begin{itemize}
    \item author: Name des Autors der Bachelorarbeit
    \item resultSummary: Zusammenfassung des Ergebnisses
    \item totalPlagiarismCoverage: Prozentualer Anteil aller Plagiate am gesamten Dokument
    \item plagiarismSources: Liste mit Quellen die als Plagiat erkannt wurden
    \begin{itemize}
        \item isbn: ISBN Nummer der Quelle
        \item plagiarismLength: Prozentualer Anteil der Plagiate dieser Quelle an der Bachelorarbeit
        \item occurrences: Liste mit Fundstellen von Plagiaten aus dieser Quelle
        \begin{itemize}
            \item startChar: Startposition des Plagiats in der Bachelorarbeit
            \item endChar: Endposition des Plagiats in der Bachelorarbeit
            \item certainty: Sicherheit mit der das Plagiat erkannt wurde
        \end{itemize}
    \end{itemize}
\end{itemize}


\section{Schnittstelle zu KI}\label{sec:schnittstelle-ki}

\begin{table}[H]
    \label{tab:schnittstelle-ki}
    \begin{tabularx}{\textwidth}{|l|X|}
        \hline
        \textbf{Art}         & Datenschnittstelle von/zu externem System \\
        \hline
        \textbf{Beschreibung} & Diese Schnittstelle wird von AiPlagiarismChecker verwendet, um einen Prompt an das KI-System zu übertragen.
        Der Prompt basiert dabei auf den von BachelorChecker gesendeten Daten. \\
        \hline
        \textbf{Komplexität} & einfach                                   \\
        \hline
        \textbf{Hinweise zur Verwendung} & Schnittstellen zu chatbasierten KIs sind in der Regel nicht kostenfrei, es fällt also für jede Anfrage eine Gebühr an.
        Die Verfügbarkeit der chatbasierten KI kann nicht durch AiPlagiarismChecker garantiert bzw. sichergestellt werden.
        Es besteht die Möglichkeit, dass das KI-System den übertragenen Prompt nicht interpretieren oder AiPlagiarismChecker die Antwort der KI nicht verarbeiten kann.
        In diesen beiden Fällen kann kein Ergebnis erzeugt werden. \\
        \hline
    \end{tabularx}
    \caption {Schnittstelle zu KI}
\end{table}