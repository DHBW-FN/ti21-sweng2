\chapter{Glossar}\label{ch:glossar}


\section*{Bachelorarbeit}\label{sec:glossar_bachelorarbeit}
\begin{table}[H]
    \label{tab:glossar_bachelorarbeit}
    \begin{tabularx}{\textwidth}{|l|X|}
        \hline
        \textbf{Aspekt}                  & \textbf{Beschreibung}                                                                                               \\
        \hline
        Synonyme                         & Abschlussarbeit, Bachelor-Thesis                                                                                    \\
        \hline
        Definition                       & Eine Bachelorarbeit ist eine wissenschaftliche Arbeit, die als Abschluss eines Bachelorstudiengangs verfasst wird. \\
        \hline
        Ähnliche/Gegensätzliche Begriffe & Masterarbeit (Arbeit zur Erlangung des Master-Grades)                                                                \\
        \hline
        Einschränkungen                  & keine                                                                                                               \\
        \hline
        Ansprechpartner                  & Philipp Küst                                                                                                        \\
        \hline
        Status                           & final                                                                                                               \\
        \hline
        Änderungen                       & 04.12.2023: erstellt                                                                                                \\
        \hline
    \end{tabularx}
    \caption{Glossareintrag: Bachelorarbeit}
\end{table}


\section*{Chatbasierte KI}\label{sec:glossar_chatbasierte_ki}
\begin{table}[H]
    \label{tab:glossar_chatbasierte_ki}
    \begin{tabularx}{\textwidth}{|l|X|}
        \hline
        \textbf{Aspekt}                  & \textbf{Beschreibung}                                                                                              \\
        \hline
        Synonyme                         & Dialogsystem, Chatbot, Conversational AI                                                                           \\
        \hline
        Definition & Chatbasierte KIs sind ein Teilgebiet des Überbegriffs KI.
        Diese sind darauf ausgelegt, mit Benutzern in natürlicher Sprache zu kommunizieren.
        Sie können dabei verschiedene Technologien verwenden, darunter maschinelles Lernen, natürliche Sprachverarbeitung (NLP) und oft auch neuronale Netzwerke. \\
        \hline
        Ähnliche/Gegensätzliche Begriffe & Chatbot (Computersysteme, die mit natürlicher Sprache funktionieren, allerdings dabei nicht zwingend KI verwenden) \\
        \hline
        Einschränkungen                  & keine                                                                                                              \\
        \hline
        Ansprechpartner                  & Philipp Küst                                                                                                       \\
        \hline
        Status                           & final                                                                                                              \\
        \hline
        Änderungen                       & 04.12.2023: erstellt                                                                                               \\
        \hline
    \end{tabularx}
    \caption{Glossareintrag: Chatbasierte KI}
\end{table}


\section*{JSON (JavaScript Object Notation)}\label{sec:glossar_json}
\begin{table}[H]
    \label{tab:glossar_json}
    \begin{tabularx}{\textwidth}{|l|X|}
        \hline
        \textbf{Aspekt}                  & \textbf{Beschreibung}                                  \\
        \hline
        Synonyme                         & -                                                      \\
        \hline
        Definition & JSON ist ein schlankes Datenaustauschformat, das einfach von Menschen gelesen und geschrieben werden kann und für Maschinen leicht zu analysieren und zu generieren ist.
        Es basiert auf einer Untermenge der JavaScript-Programmiersprache und bietet universelle Datenstrukturen wie Name/Wert-Paare und geordnete Listen von Werten. \\
        \hline
        Ähnliche/Gegensätzliche Begriffe & XML \& YAML (beides sind andere Datenaustauschformate) \\
        \hline
        Einschränkungen                  & keine                                                  \\
        \hline
        Ansprechpartner                  & Philipp Küst                                           \\
        \hline
        Status                           & final                                                  \\
        \hline
        Änderungen                       & 04.12.2023: erstellt                                   \\
        \hline
    \end{tabularx}
    \caption{Glossareintrag: JSON}
\end{table}


\section*{KI (Künstliche Intelligenz)}\label{sec:glossar_ki}
\begin{table}[H]
    \label{tab:glossar_ki}
    \begin{tabularx}{\textwidth}{|l|X|}
        \hline
        \textbf{Aspekt}                  & \textbf{Beschreibung}                                                                                                                                                           \\
        \hline
        Synonyme                         & AI (Artificial Intelligence), KI-System, intelligente Maschine                                                                                                                  \\
        \hline
        Definition & Künstliche Intelligenz (KI) ist ein Teilgebiet der Informatik.
        Sie imitiert menschliche kognitive Fähigkeiten, indem sie Informationen aus Eingabedaten erkennt und sortiert.
        Diese Intelligenz kann auf programmierten Abläufen basieren oder durch maschinelles Lernen erzeugt werden.
        (\url{https://www.iks.fraunhofer.de/de/themen/kuenstliche-intelligenz.html}) \\
        \hline
        Ähnliche/Gegensätzliche Begriffe & Maschinelles Lernen (Teilgebiet der Künstlichen Intelligenz), Data Science (Extraktion von Wissen und Erkenntnissen aus Daten; imitiert nicht zwingend menschliche Intelligenz) \\
        \hline
        Einschränkungen                  & keine                                                                                                                                                                           \\
        \hline
        Ansprechpartner                  & Philipp Küst                                                                                                                                                                    \\
        \hline
        Status                           & final                                                                                                                                                                           \\
        \hline
        Änderungen                       & 04.12.2023: erstellt                                                                                                                                                            \\
        \hline
    \end{tabularx}
    \caption{Glossareintrag: KI}
\end{table}


\section*{Nicht-gesperrtes PDF}\label{sec:glossar_nicht_gesperrtes_pdf}
\begin{table}[H]
    \label{tab:glossar_nicht_gesperrtes_pdf}
    \begin{tabularx}{\textwidth}{|l|X|}
        \hline
        \textbf{Aspekt}                  & \textbf{Beschreibung}                                                                                                                  \\
        \hline
        Synonyme                         & Zugängliche PDF-Datei, Offene PDF                                                                                                      \\
        \hline
        Definition & Ein nicht-gesperrtes PDF ist eine PDF-Datei, die nicht durch Passwörter geschützt ist und ohne Einschränkungen geöffnet und gelesen werden kann.
        (\url{https://helpx.adobe.com/de/acrobat/using/securing-pdfs-passwords.html}) \\
        \hline
        Ähnliche/Gegensätzliche Begriffe & Gesperrtes PDF (ein durch Passwort geschütztes PDF, bei dem der User nur durch Eingabe des richtigen Passwortes die Datei öffnen kann) \\
        \hline
        Einschränkungen                  & keine                                                                                                                                  \\
        \hline
        Ansprechpartner                  & Philipp Küst                                                                                                                           \\
        \hline
        Status                           & final                                                                                                                                  \\
        \hline
        Änderungen                       & 04.12.2023: erstellt                                                                                                                   \\
        \hline
    \end{tabularx}
    \caption{Glossareintrag: Nicht-gesperrtes PDF}
\end{table}


\section*{Plagiat}\label{sec:glossar_plagiat}
\begin{table}[H]
    \label{tab:glossar_plagiat}
    \begin{tabularx}{\textwidth}{|l|X|}
        \hline
        \textbf{Aspekt}                  & \textbf{Beschreibung}                        \\
        \hline
        Synonyme                         & Urheberrechtsverletzung, geistiger Diebstahl \\
        \hline
        Definition & Ein Plagiat ist die unrechtmäßige Aneignung von Gedanken, Ideen o.Ä. eines anderen auf künstlerischem oder wissenschaftlichem Gebiet und ihre Veröffentlichung bzw. der Diebstahl geistigen Eigentums.
        (\url{https://www.duden.de/rechtschreibung/Plagiat}) \\
        \hline
        Ähnliche/Gegensätzliche Begriffe & Eigenleistung (Gegenteil eines Plagiats)     \\
        \hline
        Einschränkungen                  & keine                                        \\
        \hline
        Ansprechpartner                  & Philipp Küst                                 \\
        \hline
        Status                           & final                                        \\
        \hline
        Änderungen                       & 04.12.2023: erstellt                         \\
        \hline
    \end{tabularx}
    \caption{Glossareintrag: Plagiat}
\end{table}


\section*{Prompt}\label{sec:glossar_prompt}
\begin{table}[H]
    \label{tab:glossar_prompt}
    \begin{tabularx}{\textwidth}{|l|X|}
        \hline
        \textbf{Aspekt}                  & \textbf{Beschreibung} \\
        \hline
        Synonyme                         & -                     \\
        \hline
        Definition & Ein Prompt enthält Anweisungen und/oder Befehle auf deren Grundlage eine (chatbasierte) KI Aufgaben bzw. Aktionen ausführt.
        Wird ein Prompt ausgeführt, analysiert die KI den Kontext und verwendet die seine trainierten Algorithmen, um eine Antwort zu generieren.
        Für gewöhnlich werden Prompts in natürlicher Sprache verfasst, sie können aber auch in speziellen Programmiersprachen verfasst werden. \\
        \hline
        Ähnliche/Gegensätzliche Begriffe & (Arbeits-)Anweisung   \\
        \hline
        Einschränkungen                  & keine                 \\
        \hline
        Ansprechpartner                  & Philipp Küst          \\
        \hline
        Status                           & final                 \\
        \hline
        Änderungen                       & 10.12.2023: erstellt  \\
        \hline
    \end{tabularx}
    \caption{Glossareintrag: Prompt}
\end{table}


\section*{REST (Representational State Transfer)}\label{sec:glossar_rest}
\begin{table}[H]
    \label{tab:glossar_rest}
    \begin{tabularx}{\textwidth}{|l|X|}
        \hline
        \textbf{Aspekt}                  & \textbf{Beschreibung}                                                                                 \\
        \hline
        Synonyme                         & -                                                                                                     \\
        \hline
        Definition                       & REST ist ein Architekturstil für Netzwerkanwendungen, der 2000 von Dr. Roy Fielding eingeführt wurde. \\
        \hline
        Ähnliche/Gegensätzliche Begriffe & SOAP \& GraphQL (beides andere Ansätze eine API zu designen)                                          \\
        \hline
        Einschränkungen                  & keine                                                                                                 \\
        \hline
        Ansprechpartner                  & Philipp Küst                                                                                          \\
        \hline
        Status                           & final                                                                                                 \\
        \hline
        Änderungen                       & 04.12.2023: erstellt                                                                                  \\
        \hline
    \end{tabularx}
    \caption{Glossareintrag: REST}
\end{table}