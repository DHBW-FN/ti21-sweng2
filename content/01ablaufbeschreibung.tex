\chapter{Ablaufbeschreibung der Systemidee}\label{ch:ablaufbeschreibung}

Es wird vorausgesetzt, dass die Daten vom BachelorChecker an die \ac{REST}-Schnittstelle des \ac{AiPC} übergeben werden.
Die Daten bestehen hierbei einerseits aus der Bachelorabeit in Form einer nicht gesperrten \ac{PDF}-Datei und den zusätzlichen Informationen in Form einer \ac{JSON}-Datei.

Ausgehend vom \ac{API} des \ac{AiPC} werden die Daten an die verarbeitende Komponente weitergeleitet.
Dort werden die Daten validiert, das heißt, es wird überprüft, ob es sich tatsächlich um ein \ac{PDF}-Dokument handelt und ob die \ac{JSON}-Datei valide ist.
Sind die Daten valide, werden diese ausgelesen und die daraus extrahierten Daten in ein Prompt überführt.
Zusätzlich wird in dem Prompt spezifiziert, dass die Ausgabe des Ergebnisses im \ac{JSON}-Format erfolgen soll.
Das Prompt wird anschließend verwendet, um eine Anfrage an die chatbasierte \ac{KI} zu stellen.

Die \ac{KI} verarbeitet die Anfrage, führt die Plagiatsprüfung anhand der übergebenen Daten und der Literaturdatenbanken durch und gibt das Ergebnis an \ac{AiPC} zurück.

\ac{AiPC} überführt  das Ergebnis in ein \ac{JSON}-Objekt mit dem richtigen Output-Schema für die Antwort an den BachelorChecker.
Die Antwort mit dem aufbereitetem Ergebnis wird an den BachelorChecker zurückgegeben und dort verarbeitet.

Sollte die Plagiatsprüfung nicht schnell genug durchgeführt werden und damit die Antwortzeit einer \ac{HTTP}-Anfrage (Timeout) überschreiten,
so wird dem Auftraggeber eine alternative Antwort zurückgegeben, die darauf hinweist, dass die Plagiatsprüfung noch nicht abgeschlossen ist.
Der Auftraggeber erhält dabei außerdem eine Möglichkeit, den Status der Verarbeitung periodisch abzurufen.
Ist die Plagiatsprüfung beendet, so steht das Ergebnis zur Verfügung und kann abgerufen werden.